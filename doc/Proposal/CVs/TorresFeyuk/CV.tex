%% start of file `template.tex'.
%% Copyright 2006-2013 Xavier Danaux (xdanaux@gmail.com).
%
% This work may be distributed and/or modified under the
% conditions of the LaTeX Project Public License version 1.3c,
% available at http://www.latex-project.org/lppl/.


\documentclass[11pt,a4paper,sans]{moderncv}        % possible options include font size ('10pt', '11pt' and '12pt'), paper size ('a4paper', 'letterpaper', 'a5paper', 'legalpaper', 'executivepaper' and 'landscape') and font family ('sans' and 'roman')

% modern themes
\moderncvstyle{classic}                            % style options are 'casual' (default), 'classic', 'oldstyle' and 'banking'
\moderncvcolor{blue}                                % color options 'blue' (default), 'orange', 'green', 'red', 'purple', 'grey' and 'black'
%\renewcommand{\familydefault}{\sfdefault}         % to set the default font; use '\sfdefault' for the default sans serif font, '\rmdefault' for the default roman one, or any tex font name
%\nopagenumbers{}                                  % uncomment to suppress automatic page numbering for CVs longer than one page

% character encoding
\usepackage[utf8]{inputenc}                       % if you are not using xelatex ou lualatex, replace by the encoding you are using
%\usepackage{CJKutf8}                              % if you need to use CJK to typeset your resume in Chinese, Japanese or Korean

% adjust the page margins
\usepackage[scale=0.75]{geometry}
%\setlength{\hintscolumnwidth}{3cm}                % if you want to change the width of the column with the dates
%\setlength{\makecvtitlenamewidth}{10cm}           % for the 'classic' style, if you want to force the width allocated to your name and avoid line breaks. be careful though, the length is normally calculated to avoid any overlap with your personal info; use this at your own typographical risks...

\usepackage{import}

% personal data
\name{Nicolás Rodrigo}{Ezequiel Torres Feyuk}
\title{Curriculum Vitae}                               % optional, remove / comment the line if not wanted
\address{Adolfo Alsina 2550 2B, CABA}{}{}% optional, remove / comment the line if not wanted; the "postcode city" and and "country" arguments can be omitted or provided empty
\phone[mobile]{+54 911 6940 7659}                   % optional, remove / comment the line if not wanted
% \phone[fixed]{01234 123456}                    % optional, remove / comment the line if not wanted
%\phone[fax]{+3~(456)~789~012}                      % optional, remove / comment the line if not wanted
\email{ezequiel.torresfeyuk@gmail.com}                               % optional, remove / comment the line if not wanted
% \homepage{www.myname.webs.com}                         % optional, remove / comment the line if not wanted
\extrainfo{Fecha de Nacimiento: 17 Mayo 1989 - DNI: 34.650.445}
% \extrainfo{DNI: 34.650.445}
%\photo[64pt][0.4pt]{picture}                       % optional, remove / comment the line if not wanted; '64pt' is the height the picture must be resized to, 0.4pt is the thickness of the frame around it (put it to 0pt for no frame) and 'picture' is the name of the picture file
%\quote{Some quote}                                 % optional, remove / comment the line if not wanted

% to show numerical labels in the bibliography (default is to show no labels); only useful if you make citations in your resume
%\makeatletter
%\renewcommand*{\bibliographyitemlabel}{\@biblabel{\arabic{enumiv}}}
%\makeatother
%\renewcommand*{\bibliographyitemlabel}{[\arabic{enumiv}]}% CONSIDER REPLACING THE ABOVE BY THIS

% bibliography with mutiple entries
%\usepackage{multibib}
%\newcites{book,misc}{{Books},{Others}}
%----------------------------------------------------------------------------------
%            content
%----------------------------------------------------------------------------------
\begin{document}
%\begin{CJK*}{UTF8}{gbsn}                          % to typeset your resume in Chinese using CJK
%-----       resume       ---------------------------------------------------------
\makecvtitle

\small{Estudiante de Ingeniería Informática actualmente completando el último año de la carrera de grado. Perseverante y proactivo, es de mi interés trabajar en equipos interdisciplinarios que permitan aprovechar los conocimientos adquiridos durante el transcurso de mi carrera profesional.}

\section{Trabajos Previos}

\vspace{6pt}

\begin{itemize}

\item{\cventry{Agosto 2009--Septiembre 2012}{Colaborador Universitario 66.02 Laboratorio}{Facultad de Ingeniería}{San Telmo}{}{\vspace{3pt}Como colaborador, mi rol consistió en dar apoyo a los estudiantes a la hora de adquirir la teoría básica en la rama de la metrología así como en la manipulación de los dispositivos de medición utilizados en la materia (Multímetro, Amperímetro, Osciloscopio, Contador, etc.). En los últimos años tuve la posibilidad de preparar clases y parciales con la supervisión del coordinador de la materia.}}

\vspace{6pt}

\item{\cventry{July 2012--June 2013}{Firmware and Software Developer}{Veccsa S.A.}{V. Urquiza}{}{\vspace{3pt}Mi trabajo consistió en mantener aplicaciones para la creación de informes médicos y procesamiento de señales médicas de dispositivos tales como Holters Cardíacos y Ergometrías. Durante este tiempo, también tuve como responsabilidad el mantenimiento del firmware de los dispositivos médicos, los cuales consistían en el sensado y almacenamiento de las señales obtenidas de los pacientes.}}

\vspace{6pt}

\item{\cventry{Septiembre 2013--Actualidad}{Software Developer}{Intraway S.R.L.}{V. Urquiza}{}{\vspace{3pt}Mi trabajo en Intraway consiste en el mantenimiento de diferentes aplicaciones de software dentro del estandar \textbf{DOCSIS}. En particular, poseo ownership del componente TFTP, el cual se encarga de confeccionar los archivos de configuración dentro de la red de los \textit{ISPs (Internet Service Providers)} requeridos por Cablemodems y MTAs.}}

\end{itemize}

\section{Education}

\vspace{5pt}

\subsection{Academic Qualifications}

\vspace{5pt}

\begin{itemize}

\item{\cventry{2007-Actualidad}{Ingeniería en Informática}{Facultad de Ingeniería}{San Telmo}{Principios de 2017}{}}

\item{\cventry{2003-2006}{Título Técnico en Informática Profesional y Personal}{Colegio Parroquial Juan Beat XXIII}{Ramos Mejía}{Polimodal}{}}  % arguments 3 to 6 can be left empty

\item{\cventry{1994--2002}{Nivel Primario}{Colegio María Mazzarello}{San Justo}{}{}}

\end{itemize}

\vspace{2pt}

\subsection{Proyectos}

\vspace{5pt}

\begin{itemize}

\item{\textbf{DOCSIS Device Simulator:} \textit{'Desarrollo de Simulador de Dispositivos'}

\vspace{3pt}

\small{Estuve trabajando durante más de un año en el diseño e implementación de un simulador masivo de dispositivos. El mismo permite emular el flujo de diferentes dispositivos dentro del estandar \textbf{DOCSIS} enviando paquetes a través de la red mediante el uso de \textit{Raw Sockets} para simular el uso de IPs ficticias. Algunos de los protocolos que el mismo soporta son \textit{DHCP, TFTP, SNMP, ToD, MGCP y COPS}. El flujo de cada dispositivo es configurable, y se han realizado pruebas exitosas con más de 1 millón de dispositivos. }}
\end{itemize}

\section{Conocimientos Técnicos}

\vspace{6pt}

\begin{itemize}

\item \textbf{Lenguajes de Programación:} Semi-Senior en: C, C++, Python \\ Conocimientos en otros lenguajes: Java, JavaScript, PL/SQL, Matlab, LaTeX, XML, bash scripting.

\vspace{6pt}

\item \textbf{Habilidades en Software:} Conocimientos avanzado en IPCs y programación con concurrente (Threads y Sockets). Conocimientos intermedios en Sistemas Distribuídos. 

\vspace{6pt}

\item \textbf{Other:} Conocimientos básicos en Procesamiento de Señales Determínisticas y Estocásticas, diseño y armado de circuitos integrados. Experiencia en programación con microntroladores (Arduino, PICs, etc.)

\end{itemize}

%\section{Interests and extra-curricular activity}

%\vspace{6pt}

%\begin{itemize}

%\item{I was a "fresher representative" in my 2nd and 3rd years of university, this required me to guide, look after, and ensure that a particular flat of first years have a good time in their first week, and feel consoled in what for most of them is there first time living away from home. We were responsible for the safety and wellbeing of the group of first years during the first week, and during this time I made good friends with all of them.}

%\vspace{6pt}

%\item{I am a member of a number of university societies. I was also the vice president and co-founder of the flash mob society. My roles in this included recruiting members, in which during "fresher's fair" we enlisted over 200 new members. This was regarded as very successful, considering other societies averaged around 50. I also appeared in an interview on the university television station, set up a society bank account, and helped organise the events. One of these events was featured in the local newspaper.}

%\vspace{6pt}

%\item{I am also an avid hiker, having completed the national 3 peaks challenge last summer. Other interest include guitar, which I am self-taught, and home brewing.}

%\end{itemize}

%\section{References}

%\vspace{6pt}
 
%\begin{itemize}

%\item{Up to 4 references available on request}

%\end{itemize}

% Publications from a BibTeX file without multibib
%  for numerical labels: \renewcommand{\bibliographyitemlabel}{\@biblabel{\arabic{enumiv}}}% CONSIDER MERGING WITH PREAMBLE PART
%  to redefine the heading string ("Publications"): \renewcommand{\refname}{Articles}
\nocite{*}
\bibliographystyle{plain}
\bibliography{publications}                        % 'publications' is the name of a BibTeX file

% Publications from a BibTeX file using the multibib package
%\section{Publications}
%\nocitebook{book1,book2}
%\bibliographystylebook{plain}
%\bibliographybook{publications}                   % 'publications' is the name of a BibTeX file
%\nocitemisc{misc1,misc2,misc3}
%\bibliographystylemisc{plain}
%\bibliographymisc{publications}                   % 'publications' is the name of a BibTeX file

%-----       letter       ---------------------------------------------------------

\end{document}


%% end of file `template.tex'.

